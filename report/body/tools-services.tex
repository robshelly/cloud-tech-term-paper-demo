\section{Tools and Services}
	This paper will utilize a number of different tools and services to implement CI/CD for the sample web app. Part of the objective of this paper is to gain an understanding of each of these tools and explore how they can be integrated to implement the desired workflow in a secure manner.

	\subsection{Jenkins} Jenkins is an automates build server, often used for CI/CD. It features a large library of plugins for integrating the server with various service such as GitHub and Docker Hub which will be explored in this paper. It also simple to setup and features a web app interface for easy configuration. IT also boasts ans REST API and a CLI for interfacing with the server \citep{jenkins}. These methods won't be explored in this paper however. The range of plugins available makes Jenkins and ideal build server for this project.
	
	\subsection{ECS} ECS is AWS' container management service. It allows running Docker images on EC2 instances.  Application images can be scaled and load balance across multiple containers and multiple instances. Through the use of task definitions and service ECS abstract much the of the management that required if deploying a scalable application using self managed docker images \citep{ecs}.  ECS' ability to deploy Docker images makes it an ideal candidate for this workflows as Jenkins will be build the web app as Docker images.
	
	\subsection{IAM} Identity Access Management is AWS' service for providing secure methods of to granting users and infrastructure permissions to communicate and integrate with other resources \citep{iam}. As the Jenkins server will be creating/modifying resources within AWS, ot will need a set a credentials to do so. This may pose a security threat. However, with the use of IAM, a user can be created for Jenkins with it's own set of access credentials. The user can be given bespoke policies which allow it to perform only the tasks it actions to perform. 